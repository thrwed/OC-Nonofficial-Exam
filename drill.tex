\documentclass[12pt,a4paper]{article}
\usepackage{ctex}
\usepackage{amsmath,amssymb}
\usepackage{graphicx}
\usepackage{enumitem}
\usepackage{titlesec}
\usepackage{fancyhdr}
\usepackage{lastpage}
\usepackage{geometry}
\usepackage{ulem}

% 页面设置
\geometry{left=2cm,right=2cm,top=2.5cm,bottom=2.5cm}
\pagestyle{fancy}
\fancyhf{}
\fancyhead[C]{古代漢語能力測定}
\fancyfoot[C]{第 \thepage\ 页 共 \pageref{LastPage} 页}

% 设置段落间距
\setlength{\parskip}{0.5em}
\setlength{\parindent}{0em}

% 自定义标题格式
\titleformat{\section}{\centering\Large\bfseries}{第\chinese{section}部分}{1em}{}
\titleformat{\subsection}{\large\bfseries}{\thesubsection}{1em}{}

% 选择题选项格式
\newlist{choices}{enumerate}{1}
\setlist[choices]{label=\Alph*., leftmargin=*, itemsep=0.5ex}

\begin{document}

\begin{center}
    \LARGE\textbf{古漢語知識能力測定(試行)}
    
    \vspace{1em}
    
    \large(考試時間:2小時 ;試卷滿分:150分)
\end{center}

\vspace{2em}

\noindent\textbf{惟曰:}
\begin{enumerate}[leftmargin=*, itemsep=0.5ex]
    \item 本卷務必以繁體漢字及標準的上古漢語作答,夫其欲援簡體漢文中之新概念並框之以括號,簡體是寫
    \item 今語言測定蓋聞,言,讀,作四項,然而如今我衆不能聞古語,亦不能言之,故察之以音韻
\end{enumerate}

\vspace{2em}

\section{音韻(35分)}

\subsection{上古音(共5小题,19分)}

\vspace{1em}

\begin{enumerate}[leftmargin=*, label=\textbf{\arabic*.}]
    \item 列上古音诸假设(3分)
    

    \item 舉一上古詞綴,加以論之(3分)
    
   
    \item 胡謂日居月諸(3分)
 
    
    \item 王力三十部何也(4分)
    
    \vspace{8em}
    
    \item 焉于四聲離析諸上古三聲(6分)
    
    \vspace{10em}
\end{enumerate}

\subsection{中古音Ⅱ(16分)}


\vspace{1em}

\begin{enumerate}[leftmargin=*, label=\textbf{\arabic*.}, start=6]
    \item 中古漢語有( )聲調(3分)
    
    \begin{choices}
        \item 8
        \item 9
        \item 4
        \item 6
    \end{choices}
    列舉之:
    \item 四等及介音(3分)
    
    \item 何爲脂微分韻(4分)
    
    \vspace{8em}
    
    \item 簡論重鈕(6分)
    
    \vspace{10em}
\end{enumerate}
\vspace{1em}

\begin{enumerate}[leftmargin=*, label=\textbf{\arabic*.}, start=10]

    
\section{文字(35分)}

    \item 漢字有( )書(3分)
    
    \begin{choices}
        \item 3
        \item 6
        \item 9
        \item 12
    \end{choices}
  列舉之:  
    \item 孰謂假借(3分)
 
\end{enumerate}


\section{訓詁(20分)}

\begin{enumerate}[leftmargin=*, label=\textbf{\arabic*.}, start=15]
    \item 黃侃之謂訓詁四法何也者(3分)
   
    \item “猶x也”即謂斯字(6分)
    
    \vspace{10em}
\end{enumerate}
\subsection{默寫(本题共1小题,6分)}

\begin{enumerate}[leftmargin=*, label=\textbf{\arabic*.}, start=17]
    \item 補寫經義及句讀。(6分)
    
    (1) 寧武子邦
    
    (2) 離婁之明公
    
    (3) 絕智棄辨民
\end{enumerate}

\subsection{釋文(本题共2大题,23分)}


建武三年十二月癸丑朔乙卯,都鄉嗇夫宮以廷所移甲渠候書召恩詣鄉。先以「證財物故不以實,臧伍佰以上,辭已定,滿參日而不更言請者,以辭所出入,罪反罪之」律辨告,乃爰書驗問。恩辭曰:潁川昆陽市南里,年陸拾陸歲,姓寇氏。去年十二月中,甲渠令史華商、尉史周育當為候粟君載魚之觻得賣。商、育不能行。商即出牛壹頭,黃、特、齒捌歲,平賈直陸拾石,與它穀拾伍石,為穀柒拾伍石;育出牛壹頭,黑、特、齒伍歲,平賈直陸拾石,與它穀卌石,凡為穀佰石,皆予粟君,以當載魚就直。時,粟君借恩為就,載魚伍仟頭到觻得,賈直:牛壹頭、穀貳拾柒石,約為粟君賣魚沽出時行錢卌萬。時,粟君以所得商牛黃、特、齒捌歲,以穀貳拾柒石予恩顧就直。後貳、參日當發,粟君謂恩曰:「黃牛微瘦,所得育牛黑、特,雖小,肥,賈直俱等耳,擇可用者持行。」恩即取黑牛去,留黃牛,非從粟君借牛。恩到觻得賣魚盡,錢少,因賣黑牛,並以錢參拾貳萬付粟君妻業,少捌萬。恩以大車半軸壹,直萬錢;羊韋壹枚為橐,直參仟;大笥壹合,直仟;壹石去盧壹,直陸佰;索貳枚,直仟;皆置業車上。與業俱來還,到第參置,恩糴大麥貳石付業,直陸仟;又到北部,為業買肉拾斤,直穀壹石,石參仟。凡並為錢貳萬肆仟陸佰,皆在粟君所。恩以負粟君錢,故不從取器物。又恩子男欽,以去年十二月貳拾日為粟君捕魚,盡今年正月、閏月、貳月,積作參月拾日,不得賈直。時,市庸平賈大男日貳斗,為穀貳拾石。恩居觻得付業錢時,市穀決石肆仟,以欽作賈穀拾參石捌斗伍升,直觻得錢伍萬伍仟肆,凡為錢捌萬,用償所負錢畢。恩當得欽作賈餘穀陸石壹斗伍升,付恩從觻得自食,為業將車到居延,積行道貳拾餘日,不計賈直。時,商、育皆平牛直陸拾石與粟君,因以其賈予恩,已決,恩不當與粟君牛,不相當穀貳拾石。皆證也,如爰書。建武三年十二月癸丑朔戊辰,都鄉嗇夫宮以廷所移甲渠候書召恩詣鄉。先以證財物故不以實,臧五百以上,辭以定,滿三日而不更言請者,以辭所出入,罪反罪之律辨告,乃爰書驗問。恩辭曰:潁川昆陽市南里,年六十六歲,姓寇氏。去年十二月中,甲渠令史華商、尉史周育當為候粟君載魚之觻得賣。商、育不能行。商即出牛一頭,黃、特、齒八歲,平賈值六十石,與它穀十五石,為穀七十五石。育出牛一頭,黑、特、齒五歲,平賈值六十石,與它穀卌石,凡為穀百石,皆予粟君,以當載魚就直。時,粟君借恩為就,載魚五千頭到觻得,賈直:牛一頭、穀廿七石,[約]為粟君賣魚沽出時行錢卌萬。時,粟君以所得商牛黃、特、齒八歲,以穀廿七石予恩顧對直。後二、三日當發,粟君謂恩曰:黃牛微庾,所將(得)育牛黑、特,雖小,肥,賈直俱等耳,擇可用者持行。恩即取黑牛去,留黃牛,非從粟君借牛。恩到觻得賣魚盡,錢少,因賣黑牛,并以錢卅二萬付粟君妻業,少八歲(萬)。恩以大車半磨軸一,直萬錢;羊韋一枚為橐,直三千;大笥一合,直千;一石去盧一,直六百,庫索二枚,直千,皆置業車上。與業俱來還,到弟(第)三置,為業籴大麥二石。凡為穀三石,錢萬五千六百,皆在業所。恩與業俱來到居延後,恩欲取軸、器物去。粟君謂恩:汝負我錢八萬,欲持器物?怒。恩不取器物去。又恩子男欽,以去年十二月廿日為粟君捕魚,盡今年正月、閏月、二月,積作三月十日,不得賈直。時,市庸平賈大男日二斗,為穀廿石。恩居觻得付業錢時,市穀決石四千。并以欽作賈穀,當所負粟君錢畢。恩又從觻得自食為業將車,坐斬來到居延,積行道廿餘日,不計賈直。時,商、育皆平直牛六十石與粟君,因以其賈與恩,牛已決,不當予粟君牛,不相當穀廿石。皆證也,如爰書。建武三年十二月癸丑朔辛未,都鄉嗇夫宮敢言之。廷移甲渠候書曰:去年十二月中,取客寇恩為就,載魚五千頭到觻得,就賈用牛一頭,穀廿七石,恩願沽出時行錢卌萬。以得卅十二萬。又借牛一頭以為犅,因賣,不肯歸以所得就直牛,償不相當廿石。書到。驗問。治決言。前言解廷郵書曰:恩辭不與候書相應,疑非實。今候奏記府,願詣鄉爰書自證。府錄:令明處更詳驗問。治決言。謹驗問,恩辭,不當與粟君牛,不相當穀廿石,又以在粟君所器物直錢萬五千六百,又為粟君買肉,糴三石,又子男欽為粟君作賈直廿石,皆盡償所負粟君錢畢。粟君用恩器物幣(敝)敗,今欲歸恩,不肯受。爰書自證。寫移爰書,叩頭死罪死罪敢言之。
右爰書
十二曰己卯,居延令守臣移甲渠候官。候所責男子寇恩事,鄉口辭,爰書自證。寫移書到口口口口口辭,爰書自證。須以政不直者法亟報。 如律令。 掾黨、守令史賞。
建武三年十二月候粟君所責寇恩事


\vspace{1em}

\begin{enumerate}[leftmargin=*, label=\textbf{\arabic*.}, start=18]
    \item 簡言其事(6分)
    
    \vspace{10em}
    
    \item 孰冤,所冤何也(4分)
    
    \vspace{6em}
    
曰古有赤鳩,集于湯之屋,湯射之獲之,乃命小臣曰:“脂羹之,我其享之。”湯往囗。
小臣既羹之,湯后妻紝巟謂小臣曰:“嘗我於爾羹。”
小臣弗敢嘗,曰:“后其殺我。”
紝巟謂小臣曰:“爾不我嘗,吾不亦殺爾?”
小臣自堂下授紝巟羹。紝巟受小臣而嘗之,乃昭然,四荒之外,亡不見也;小臣受其餘而嘗之,亦昭然四海之外,亡不見也。
湯返廷,小臣饋。湯怒曰:“孰調吾羹?”
小臣懼,乃逃于夏。
湯乃魅之,小臣乃眛而寢於路,視而不能言。
眾烏將食之,巫烏曰:“是小臣也,不可食也。夏后有疾,將撫楚,于食其祭。”
眾烏乃訊巫烏曰:“夏后之疾如何?”
巫烏乃言曰:“帝命二黃蛇與二白兔居后之寢室之棟,其下舍后疾,是使后疾疾而不知人。帝命后土為二陵屯,共居后之床下,其上刺后之體,是使后之身疴蠚,不可及于席。”
眾烏乃往。巫烏乃度小臣之喉胃,小臣乃起而行,至於夏后。
夏后曰:“爾惟誰?”
小臣曰:“我天巫。”
夏后乃訊小臣曰:“如爾天巫,而知朕疾?”
小臣曰:“我知之。”
夏後曰:“朕疾如何?”
小臣曰:“帝命二黃蛇與二白兔,居后之寢室之棟,其下舍后疾,是使后棼棼眩眩而不知人。帝命后土為二蓤屯,共居后之牀下,其上刺后之身,是使后昏亂甘心。后如撤屋,殺黃蛇與白兔,發地斬陵,后之疾其廖。”
夏后乃從小臣之言,撤屋,殺二黃蛇與一白兔;乃發地,有二蓤廌(屯),乃斬之。其一白兔不得,是始為埤丁諸屋,以御白兔。   
    
    \item 簡論梗概(5分)
    
    \vspace{8em}
    
    \item 小臣構陷於何事(3分)
    
    \vspace{4em}
    
    \item 是湯及后其謀在先乎 (5分)
    
    \vspace{8em}
\end{enumerate}

\section{作文(60分)}

\begin{enumerate}[leftmargin=*, label=\textbf{\arabic*.}, start=23]
    \item 參考(60分)
    
    \textbf{議論:}
    \begin{itemize}[leftmargin=2em, itemsep=0.5ex, parsep=0.5ex]
        \item 真理之於科學也
        \item 欲和平即欲正義乎
        \item 人種异也若之何
    \end{itemize}
    
    \textbf{制藝:}
    \begin{itemize}[leftmargin=2em, itemsep=0.5ex, parsep=0.5ex]
        \item 我克協我友,今惟民遠邦歸志
        \item 子曰:此道之美也,莫之御也
        \item 吾先君必將相乳,以定鄭邦之社稷
        \item 寧武子邦
        \item 君夫人陽貨欲
    \end{itemize}
    
    \textbf{策論:}
    \begin{itemize}[leftmargin=2em, itemsep=0.5ex, parsep=0.5ex]
        \item 隋唐胡漢同爲編氓,然今分之論
        \item 學者不窺其取材所由,而徒校其成器所至論
        \item 元元之倒懸,故夫相責他人,亟宜驳正謬論,安民定邦策
    \end{itemize}
    
    \vspace{1em}
    
    \textbf{惟曰:}凡欲寫駢賦及詩詞者,必先註明所用之韻;欲寫制藝等文體者,必按其格式;不逾500字。
    
    \vspace{25em}
\end{enumerate}


\end{document}
